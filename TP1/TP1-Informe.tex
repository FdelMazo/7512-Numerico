\documentclass[11pt,a4paper]{article}
\usepackage[spanish]{babel}
\usepackage[utf8]{inputenc}
\usepackage[top=2cm, bottom=2cm, left=2cm, right=2cm]{geometry}
\usepackage{commath,amsmath,lastpage,float,sectsty,hyperref,graphicx,pdfpages,fancyhdr,listings,siunitx}

\sectionfont{\fontsize{12}{15}\selectfont}
\linespread{1.25} % Interlineado 1.5
\renewcommand{\rmdefault}{phv} % Arial
\renewcommand{\sfdefault}{phv} % Arial

\hypersetup{
    pdftitle={Trabajo Práctico 1},
	pdfsubject={Análisis Numérico I},
	pdfauthor={del Mazo Federico, Kristal Juan Ignacio},
}

\lstset{
  basicstyle=\ttfamily,
  columns=fullflexible,
  frame=single,
  breaklines=true,
  postbreak=\mbox{\textcolor{red}{$\hookrightarrow$}\space},
}

\fancypagestyle{enunciado}{
    \fancyhf{}
    \fancyhead[C]{Enunciado provisto por la catedra}
}

\pagestyle{fancy}
\fancyhf{}
\fancyhead[R]{Trabajo Práctico 1 - 2018c2}
\fancyhead[L]{75.12 Análisis Numérico I}
\renewcommand{\headrulewidth}{0.4pt}
\fancyfoot[L]{100029 - 99779}
\fancyfoot[R]{\thepage de \pageref*{LastPage}}
\renewcommand{\footrulewidth}{0.4pt}
\setlength{\footskip}{17pt}

\fancypagestyle{onlyheader}{
\fancyfoot{}
}

\begin{document}

\begin{titlepage}
	\hfill\includegraphics[width=6cm]{figuras/fiuba.jpg}
    \begin{center}
    \vfill
    \Huge \textbf{Trabajo Práctico 1}
    \vskip2cm
    \Large [75.12] Análisis Numérico I\\
    Segundo cuatrimestre de 2018
    \vfill
    \begin{tabular}{|l|c|r|}
	\hline
	Alumno & Padrón & Mail\\
	\hline
	\hline
	del Mazo, Federico & 100029 & delmazofederico@gmail.com\\
	\hline
	Kristal, Juan Ignacio & 99779 & kristaljuanignacio@gmail.com\\
	\hline
	\end{tabular}
    \vskip2cm
    \end{center}

    Curso 07:

    \begin{itemize}
    \item Dr Daniel Fabian Rodriguez
    \item Valeria Machiunas
    \item Federico Balzarotti
    \item Michael Portocarrero
    \end{itemize}

\end{titlepage}

\includepdf[pages=-,pagecommand={\thispagestyle{enunciado}}]{TP1-Enunciado.pdf}

\pagenumbering{gobble}
\tableofcontents
\thispagestyle{onlyheader}
\newpage

\pagenumbering{arabic}
\setcounter{page}{1}

\section{Introducción}
El trabajo práctico tiene como objetivo el cálculo y acotamiento de errores de la siguiente integral:

\[ F(\alpha,\beta) = \int_{1}^{240}  \frac{\sin{(P x)}  + \beta x^2}{\alpha x} dx \]

Siendo:
\begin{itemize}
\item \( P = \frac{\sum\limits_{padrones}^{}}{50} = \frac{99779 + 100029}{50} = 3996.16 \) exacto 
\item \( \alpha = 0.17 \) bien redondeando
\item \( \beta = 0.41 \) bien redondeando
\end{itemize}

Específicamente:

\begin{itemize}
\item Se estimará el valor de la unidad de maquina \( \mu \).
\item Se evaluará la integral con el método de trapecios compuestos teniendo con objetivo en mente que el error absoluto de truncamiento sea menor a \num{1e-5}.
\item Se calculará computacionalmente la cantidad de trapecios utilizada en el método descrito anteriormente.
\item Se calculará la condición del problema mediante perturbaciones experimentales.
\item Se estimará experimentalmente el término de estabilidad.
\item Se acotará el error total.
\end{itemize}

\section{Desarrollo}

\subsection{Estimación de \( \mu \) }

Para los cálculos del \( \mu \) se utilizó el algoritmo del ejemplo 6.4 del libro de Hernan Gonzales \cite{Gonzales}.

\subsection{Función y sus derivadas}

Siempre teniendo en cuenta los valores de \(P, \alpha, \beta \) previamente utilizados, definimos la funcion \( f(x) \) como:

\[ f(x) = \frac{\sin{(P x)}  + \beta x^2}{\alpha x} dx \]

Graficamos la función para saber un poco más de ella en la figura \ref{fig:funcion}

\begin{figure}[H]
	\makebox[\textwidth][c]{\includegraphics[width=1.2\textwidth]{figuras/funcion.jpg}}
	\caption{\(f(x)\)}
	\label{fig:funcion}
\end{figure}

De esta función calculamos sus derivadas y las graficamos en las figuras \ref{fig:funcionderivada} y \ref{fig:funcionderivada2}, para utilizar en cálculos posteriores.

\[ f'(x) = \frac{P \cos{(P x)}}{\alpha x} - \frac{ \sin{(P x)}}{\alpha x^2} + \frac{\beta}{\alpha} \]

\begin{figure}[H]
	\makebox[\textwidth][c]{\includegraphics[width=1.2\textwidth]{figuras/funcionderivada.jpg}}
	\caption{\(f'(x)\)}
	\label{fig:funcionderivada}
\end{figure}

\[ f''(x) = - \frac{2 P \cos{(P x)}}{\alpha x^2} + \frac{2 \sin{(P x)}}{\alpha x^3} - \frac{P^2 \sin{(P x)}}{\alpha x}\]

\begin{figure}[H]
	\makebox[\textwidth][c]{\includegraphics[width=1.2\textwidth]{figuras/funcionderivada2.jpg}}
	\caption{\(f''(x)\)}
	\label{fig:funcionderivada2}
\end{figure}

\subsection{Método de trapecios compuestos}

Sabemos que el error de truncamiento producido por el método de trapecios compuestos es:

\[ \epsilon_t = - \frac{{(b - a)}^3}{12 n^2} * f''(\xi) \]

Donde \(b, a\) son los limites de integración y \(n\) es la cantidad de trapecios. Como lo que queremos es acotar el error de truncamiento, debemos evaluar a la segunda derivada en su imagen máxima, es decir \(\xi = 1 \)

Por lo tanto, y ahora con el error de truncamiento al que queremos llegar, despejamos la cantidad de trapecios:

\[ n = \sqrt{ \abs{ - \frac{(b - a)^3 * f''(1)}{12 \epsilon_t}}} \]

Es con este n que se puede finalmente implementar la función de el método de los trapecios compuestos.

\subsubsection{Truncamiento de n}

Al despejar por el método de los trapecios compuestos el n necesario para tener el error deseado se notó que este valor era de tal magnitud y orden que computacionalmente carecería de sentido usarlo para cada cálculo. Es por esto que se decidió hacer un truncamiento de este valor, para poder tratarlo como es debido y en un lógico margen de tiempo. De todas formas, solo anecdóticamente, se incluye una corrida del programa con el n original.

El criterio para truncar n es el de ver como escala el cálculo de la integral respecto del valor, y luego decidir un punto de corte tratable arbitrariamente (en nuestro caso, 5 minutos). Se puede ver en el gráfico \ref{fig:calcularn} que esta es una función lineal, lo cual tiene sentido ya que lo único que adiciona computacionalmente es el ciclo definido \texttt{for} de la función, haciendolo \(\mathcal{O}(n)\), siendo n el mismo n con el que venimos tratando, redundantemente.

\begin{figure}[H]
	\makebox[\textwidth][c]{\includegraphics[width=1.2\textwidth]{figuras/calcularn.jpg}}
	\caption{Calculo de la integral para distintos n en funcion del tiempo}
	\label{fig:calcularn}
\end{figure}

\subsection{Condición del problema y término de estabilidad}

\section{Resultados}

\subsection{Estimación de \( \mu \) }

Con el algoritmo utilizado se llego al resultado \texttt{\num{1e-8}} para el \(\mu\) de precisión simple y \texttt{\num{1e-16}} para el \(\mu\) de precisión doble.

\subsection{Método de trapecios compuestos}

\subsubsection{Precisión simple}

Haciendo los calculos se vió que n es igual a \texttt{\num{2.4160e8}}, y luego se decidió truncar el número n a \texttt{5000000} para que sea un cálculo tratable. Para el n original la integral dió como resultado \texttt{\num{6.9458e+04}}, mientras que para n truncado dió

\subsubsection{Precisión doble}

Haciendo los calculos se vió que n es igual a \texttt{241403216}, y luego se decidió truncar el número n a \texttt{5000000}. Para el n original la integral dió como resultado \texttt{\num{6.9458e+04}}, mientras que para n truncado dió

\section{Conclusiones}

\newpage
\appendix
\section{Anexo I: Código Fuente}

\lstinputlisting[language=Octave,title=\texttt{mu.m}]{src/mu.m}

\newpage
\lstinputlisting[language=Octave,title=\texttt{main.m}]{src/main.m}

\newpage
\lstinputlisting[language=Octave,title=\texttt{singlemain.m}]{src/singlemain.m}

\newpage
\begin{lstlisting}[language=Octave,title=Generación de graficos]
fplot(@f, [-0.02 0.02])
fplot(@fderivada, [-0.02 0.02])
fplot(@fderivada2, [-0.02 0.02])
\end{lstlisting}

\begin{lstlisting}[language=Octave,title=Truncamiento y gráfico de n]
function y = graficar_n()
x = [1,10,100,1000,10000,100000,1000000,10000000];
y = [];
for n = x
    n
    tic;
    integral = calcular_area(n)
    y = [y,toc]; 
    printf("Tiempo = %ds\n\n",y(length(y)))   
end
plot(x,y,'o-r')
end
\end{lstlisting}

\newpage
\section{Anexo II: Resultados Numéricos}

\begin{lstlisting}[language=Octave,title=Resultados de \texttt{mu.m}]
>> mu
mu_simple =    1.0000e-08
mu_doble =    1.0000e-16
\end{lstlisting}

\begin{lstlisting}[language=Octave,title=Resultados originales sin truncamiento de n]
>> main
n =    2.4160e+08
Elapsed time is 12404.8 seconds.
ans =    6.9458e+04
\end{lstlisting}

\begin{lstlisting}[language=Octave,title=n con precisión simple]
>> main
n = 241403216
\end{lstlisting}

\begin{lstlisting}[language=Octave,title=Resultados con truncamiento de n]
>> main
n =  5000000
integral =    6.9458e+04
Elapsed time is 360.336 seconds.
>> singlemain
n =  5000000
integral =    7.0236e+04
Elapsed time is 450.23 seconds.
\end{lstlisting}

\begin{lstlisting}[language=Octave,title=Cálculos hechos para el criterio de truncamiento de n]
>> graficar_n
n =  1
integral =    6.9497e+04
Tiempo = 0.000295877s

n =  10
integral =    6.9459e+04
Tiempo = 0.000695944s

n =  100
integral =    6.9465e+04
Tiempo = 0.00530505s

n =  1000
integral =    6.9465e+04
Tiempo = 0.0516629s

n =  10000
integral =    6.9458e+04
Tiempo = 0.509583s

n =  100000
integral =    6.9458e+04
Tiempo = 5.11305s

n =  1000000
integral =    6.9458e+04
Tiempo = 51.1415s

n =  10000000
integral =    6.9458e+04
Tiempo = 599.773s
\end{lstlisting}

\newpage

\phantomsection 
\addcontentsline{toc}{section}{Bibliografía}
\renewcommand\refname{Bibliografía}
\begin{thebibliography}{9}

\bibitem{Gonzales} 
Gonzales, Hernan: 
\textit{Análisis Numérico, Primer Curso}
Buenos Aires: Nueva Librería, 2002.

\end{thebibliography}
\end{document}

